% Options for packages loaded elsewhere
\PassOptionsToPackage{unicode}{hyperref}
\PassOptionsToPackage{hyphens}{url}
%
\documentclass[
]{article}
\usepackage{amsmath,amssymb}
\usepackage{iftex}
\ifPDFTeX
  \usepackage[T1]{fontenc}
  \usepackage[utf8]{inputenc}
  \usepackage{textcomp} % provide euro and other symbols
\else % if luatex or xetex
  \usepackage{unicode-math} % this also loads fontspec
  \defaultfontfeatures{Scale=MatchLowercase}
  \defaultfontfeatures[\rmfamily]{Ligatures=TeX,Scale=1}
\fi
\usepackage{lmodern}
\ifPDFTeX\else
  % xetex/luatex font selection
\fi
% Use upquote if available, for straight quotes in verbatim environments
\IfFileExists{upquote.sty}{\usepackage{upquote}}{}
\IfFileExists{microtype.sty}{% use microtype if available
  \usepackage[]{microtype}
  \UseMicrotypeSet[protrusion]{basicmath} % disable protrusion for tt fonts
}{}
\makeatletter
\@ifundefined{KOMAClassName}{% if non-KOMA class
  \IfFileExists{parskip.sty}{%
    \usepackage{parskip}
  }{% else
    \setlength{\parindent}{0pt}
    \setlength{\parskip}{6pt plus 2pt minus 1pt}}
}{% if KOMA class
  \KOMAoptions{parskip=half}}
\makeatother
\usepackage{xcolor}
\usepackage[margin=1in]{geometry}
\usepackage{color}
\usepackage{fancyvrb}
\newcommand{\VerbBar}{|}
\newcommand{\VERB}{\Verb[commandchars=\\\{\}]}
\DefineVerbatimEnvironment{Highlighting}{Verbatim}{commandchars=\\\{\}}
% Add ',fontsize=\small' for more characters per line
\usepackage{framed}
\definecolor{shadecolor}{RGB}{248,248,248}
\newenvironment{Shaded}{\begin{snugshade}}{\end{snugshade}}
\newcommand{\AlertTok}[1]{\textcolor[rgb]{0.94,0.16,0.16}{#1}}
\newcommand{\AnnotationTok}[1]{\textcolor[rgb]{0.56,0.35,0.01}{\textbf{\textit{#1}}}}
\newcommand{\AttributeTok}[1]{\textcolor[rgb]{0.13,0.29,0.53}{#1}}
\newcommand{\BaseNTok}[1]{\textcolor[rgb]{0.00,0.00,0.81}{#1}}
\newcommand{\BuiltInTok}[1]{#1}
\newcommand{\CharTok}[1]{\textcolor[rgb]{0.31,0.60,0.02}{#1}}
\newcommand{\CommentTok}[1]{\textcolor[rgb]{0.56,0.35,0.01}{\textit{#1}}}
\newcommand{\CommentVarTok}[1]{\textcolor[rgb]{0.56,0.35,0.01}{\textbf{\textit{#1}}}}
\newcommand{\ConstantTok}[1]{\textcolor[rgb]{0.56,0.35,0.01}{#1}}
\newcommand{\ControlFlowTok}[1]{\textcolor[rgb]{0.13,0.29,0.53}{\textbf{#1}}}
\newcommand{\DataTypeTok}[1]{\textcolor[rgb]{0.13,0.29,0.53}{#1}}
\newcommand{\DecValTok}[1]{\textcolor[rgb]{0.00,0.00,0.81}{#1}}
\newcommand{\DocumentationTok}[1]{\textcolor[rgb]{0.56,0.35,0.01}{\textbf{\textit{#1}}}}
\newcommand{\ErrorTok}[1]{\textcolor[rgb]{0.64,0.00,0.00}{\textbf{#1}}}
\newcommand{\ExtensionTok}[1]{#1}
\newcommand{\FloatTok}[1]{\textcolor[rgb]{0.00,0.00,0.81}{#1}}
\newcommand{\FunctionTok}[1]{\textcolor[rgb]{0.13,0.29,0.53}{\textbf{#1}}}
\newcommand{\ImportTok}[1]{#1}
\newcommand{\InformationTok}[1]{\textcolor[rgb]{0.56,0.35,0.01}{\textbf{\textit{#1}}}}
\newcommand{\KeywordTok}[1]{\textcolor[rgb]{0.13,0.29,0.53}{\textbf{#1}}}
\newcommand{\NormalTok}[1]{#1}
\newcommand{\OperatorTok}[1]{\textcolor[rgb]{0.81,0.36,0.00}{\textbf{#1}}}
\newcommand{\OtherTok}[1]{\textcolor[rgb]{0.56,0.35,0.01}{#1}}
\newcommand{\PreprocessorTok}[1]{\textcolor[rgb]{0.56,0.35,0.01}{\textit{#1}}}
\newcommand{\RegionMarkerTok}[1]{#1}
\newcommand{\SpecialCharTok}[1]{\textcolor[rgb]{0.81,0.36,0.00}{\textbf{#1}}}
\newcommand{\SpecialStringTok}[1]{\textcolor[rgb]{0.31,0.60,0.02}{#1}}
\newcommand{\StringTok}[1]{\textcolor[rgb]{0.31,0.60,0.02}{#1}}
\newcommand{\VariableTok}[1]{\textcolor[rgb]{0.00,0.00,0.00}{#1}}
\newcommand{\VerbatimStringTok}[1]{\textcolor[rgb]{0.31,0.60,0.02}{#1}}
\newcommand{\WarningTok}[1]{\textcolor[rgb]{0.56,0.35,0.01}{\textbf{\textit{#1}}}}
\usepackage{longtable,booktabs,array}
\usepackage{calc} % for calculating minipage widths
% Correct order of tables after \paragraph or \subparagraph
\usepackage{etoolbox}
\makeatletter
\patchcmd\longtable{\par}{\if@noskipsec\mbox{}\fi\par}{}{}
\makeatother
% Allow footnotes in longtable head/foot
\IfFileExists{footnotehyper.sty}{\usepackage{footnotehyper}}{\usepackage{footnote}}
\makesavenoteenv{longtable}
\usepackage{graphicx}
\makeatletter
\def\maxwidth{\ifdim\Gin@nat@width>\linewidth\linewidth\else\Gin@nat@width\fi}
\def\maxheight{\ifdim\Gin@nat@height>\textheight\textheight\else\Gin@nat@height\fi}
\makeatother
% Scale images if necessary, so that they will not overflow the page
% margins by default, and it is still possible to overwrite the defaults
% using explicit options in \includegraphics[width, height, ...]{}
\setkeys{Gin}{width=\maxwidth,height=\maxheight,keepaspectratio}
% Set default figure placement to htbp
\makeatletter
\def\fps@figure{htbp}
\makeatother
\setlength{\emergencystretch}{3em} % prevent overfull lines
\providecommand{\tightlist}{%
  \setlength{\itemsep}{0pt}\setlength{\parskip}{0pt}}
\setcounter{secnumdepth}{-\maxdimen} % remove section numbering
\ifLuaTeX
  \usepackage{selnolig}  % disable illegal ligatures
\fi
\IfFileExists{bookmark.sty}{\usepackage{bookmark}}{\usepackage{hyperref}}
\IfFileExists{xurl.sty}{\usepackage{xurl}}{} % add URL line breaks if available
\urlstyle{same}
\hypersetup{
  pdftitle={MovieLens Report Project},
  pdfauthor={Ernesto José Hernández Navarro},
  hidelinks,
  pdfcreator={LaTeX via pandoc}}

\title{MovieLens Report Project}
\author{Ernesto José Hernández Navarro}
\date{2024-03-25}

\begin{document}
\maketitle

{
\setcounter{tocdepth}{2}
\tableofcontents
}
\hypertarget{introduction}{%
\section{Introduction}\label{introduction}}

Currently Machine Learning, one of the disciplines of artificial
intelligence, has become one of the most important tools for public and
private companies, in sectors such as health, economy, sports, traffic
and movies, which is the case we are going to be analyzing this
document.

Netflix organized a contest for a team that could create a movie
recommendation system in 2006. This algorithm developed by BellKor's
Pragmtic Chaos team in 2009 served as a reference for companies like
Amazon and offer their customers more precisely the products that they
have the most propensity to buy.

The goal of this project is to use the MovieLens dataset (which has 10
million ratings), R and RStudio, to achieve a root mean square error
(RMSE) less than 0.86490. Due to the large amount of information the edX
team has made available the code to split the data.

To help understand the data set, a data exploration will be carried out.
For the algorithm tests, the information will be divided into a training
and test set in order to validate the work done on the final model, the
limitations and possible future potential.

\hypertarget{data-exploration}{%
\section{Data exploration}\label{data-exploration}}

The data from edX data.frame has 9,000,055 rows and 6 columns, with
ratings given by a total of 69,878 unique users, with a total of 10,677
unique movies. The data has structure and does not has nulls:

\begin{verbatim}
##    userId   movieId    rating timestamp     title    genres 
##         0         0         0         0         0         0
\end{verbatim}

On the top ten most rated films we see a clearly winner with Pulp
fiction:

\begin{verbatim}
## # A tibble: 10 x 2
##    title                                                        Count_Rating
##    <chr>                                                               <int>
##  1 Pulp Fiction (1994)                                                 31362
##  2 Forrest Gump (1994)                                                 31079
##  3 Silence of the Lambs, The (1991)                                    30382
##  4 Jurassic Park (1993)                                                29360
##  5 Shawshank Redemption, The (1994)                                    28015
##  6 Braveheart (1995)                                                   26212
##  7 Fugitive, The (1993)                                                25998
##  8 Terminator 2: Judgment Day (1991)                                   25984
##  9 Star Wars: Episode IV - A New Hope (a.k.a. Star Wars) (1977)        25672
## 10 Apollo 13 (1995)                                                    24284
\end{verbatim}

The rating most used by users is 4 representing the 28.76\% of the total
ratings, the minimum is 0.5, the maximum is 5. The integer rates were
7,156,885 which represents the 79.5\% and the decimal ratings were
1,843,170 representing the 20.5\%.

\begin{figure}
\centering
\includegraphics{MovieLens-Report_files/figure-latex/- distribution-ratings-1.pdf}
\caption{Ratings distribution}
\end{figure}

\hypertarget{movies}{%
\subsection{Movies}\label{movies}}

As we know, not all the users rates the received service, and in this
case, the watched movie is not always rated, some are more rated than
others, and 126 movies where rated with once. So, there is a bias that
has to be consider on the training algorithm.

\begin{figure}
\centering
\includegraphics{MovieLens-Report_files/figure-latex/- movie-effects-1.pdf}
\caption{Distribution of rated movies in average}
\end{figure}

\hypertarget{users}{%
\subsection{Users}\label{users}}

Some users contribute more than others, and some of then a more
benevolent (high ratings) with the movies. One user made 6616 and 2 did
1059 made less than 10 ratings. The bias is clearly with the following
plot:

\begin{figure}
\centering
\includegraphics{MovieLens-Report_files/figure-latex/- user-effects-1-1.pdf}
\caption{Distribution of users by Avg. Rating}
\end{figure}

\begin{figure}
\centering
\includegraphics{MovieLens-Report_files/figure-latex/- user-effects-2-1.pdf}
\caption{\# of ratings by user}
\end{figure}

\hypertarget{genre}{%
\subsection{Genre}\label{genre}}

Movies not always has one genre, so, in the edx data set we can see that
the field ``genre'' assigns several genres and separates it with the
symbol ``\textbar{}'', and for our exploration the first one will be
take. In the following table we can see that some genres have better
ratings and a greater number of ratings:

\begin{verbatim}
## # A tibble: 20 x 3
##    genres               count rating
##    <chr>                <int>  <dbl>
##  1 Drama              3910127   3.67
##  2 Comedy             3540930   3.44
##  3 Action             2560545   3.42
##  4 Thriller           2325899   3.51
##  5 Adventure          1908892   3.49
##  6 Romance            1712100   3.55
##  7 Sci-Fi             1341183   3.4 
##  8 Crime              1327715   3.67
##  9 Fantasy             925637   3.5 
## 10 Children            737994   3.42
## 11 Horror              691485   3.27
## 12 Mystery             568332   3.68
## 13 War                 511147   3.78
## 14 Animation           467168   3.6 
## 15 Musical             433080   3.56
## 16 Western             189394   3.56
## 17 Film-Noir           118541   4.01
## 18 Documentary          93066   3.78
## 19 IMAX                  8181   3.77
## 20 (no genres listed)       7   3.64
\end{verbatim}

Drama and Comedy genre have clearly the most quantity of rates,
Documentary and IMAX are the lowest rated. Also there are seven that
don't have a genre.

\hypertarget{title}{%
\subsection{Title}\label{title}}

In the data set the column ``Title'' includes the release year of the
movie as we can see in the following table:

\begin{verbatim}
## Selecting by n
\end{verbatim}

\begin{verbatim}
## # A tibble: 10 x 2
##    title                                                            n
##    <chr>                                                        <int>
##  1 Pulp Fiction (1994)                                          31362
##  2 Forrest Gump (1994)                                          31079
##  3 Silence of the Lambs, The (1991)                             30382
##  4 Jurassic Park (1993)                                         29360
##  5 Shawshank Redemption, The (1994)                             28015
##  6 Braveheart (1995)                                            26212
##  7 Fugitive, The (1993)                                         25998
##  8 Terminator 2: Judgment Day (1991)                            25984
##  9 Star Wars: Episode IV - A New Hope (a.k.a. Star Wars) (1977) 25672
## 10 Apollo 13 (1995)                                             24284
\end{verbatim}

It looks like there is a bias on the year, on the following chart the
curve increase between 1940 and 1950, after that and starting from 1970
the curve decrease the average rating. The number of rates starts it
peak on 1990 and maximum number is on 1995.

\begin{verbatim}
## `geom_smooth()` using method = 'loess' and formula 'y ~ x'
\end{verbatim}

\includegraphics{MovieLens-Report_files/figure-latex/- year effect-1.pdf}

\begin{center}\includegraphics[width=0.75\linewidth]{MovieLens-Report_files/figure-latex/- # of rates by release year-1} \end{center}

\hypertarget{methods}{%
\subsection{Methods}\label{methods}}

\begin{verbatim}
## Warning in set.seed(2024, sample.kind = "Rounding"): non-uniform 'Rounding'
## sampler used
\end{verbatim}

\begin{verbatim}
## Joining with `by = join_by(userId, movieId, rating, timestamp, title, year,
## genres)`
\end{verbatim}

As mentioned at the introduction our goal is to reach a RMSE less than
0.86490, to do it the edx dataset needs to be split in two parts, the
training set and the test set. This together with cross-validation
methods allows to prevent over-training.

Doing the same steps learned from professor Irizarry courses the data
will be partitioned, 80\% for training and 20\% for testing using the
libraries ``caret'', ``tydiverse'' and ``dplyr''.

\hypertarget{error-loss}{%
\subsubsection{Error loss}\label{error-loss}}

The RMSE is like the standard deviation of the residual predictors. The
RMSE represent the error loss between the predicted ratings from
applying the algorithm and actual ratings in the test set. The formula
shown below, \(y_{u,i}\) is defined as the actual rating provided by a
user \(u\) for a movie \(i\), \(\hat{y}_{u,i}\) is the predicted rating
for the same, and N is the total number of user/movie combinations.

\[RMSE = \sqrt{\frac{1}{N}\sum_{u,i}\left(\hat{y}_{u,i}-y_{u,i}\right)^2}\]

\hypertarget{developing-the-algorithm}{%
\subsubsection{Developing the
algorithm}\label{developing-the-algorithm}}

The goal set for this project is to achieve a RMSE equal or less than
0.86490, then it will be about reaching the goal step by step. We need
to the fit the model with this formula:

\[Y_{u,i}=\mu+\epsilon_{u,i}\]

The common method is to use the rating mean of every user on movies.

\[Y{u,i} = \mu\]

\begin{Shaded}
\begin{Highlighting}[]
\NormalTok{mu\_hat }\OtherTok{\textless{}{-}} \FunctionTok{mean}\NormalTok{(train\_set}\SpecialCharTok{$}\NormalTok{rating)}
\FunctionTok{RMSE}\NormalTok{(test\_set}\SpecialCharTok{$}\NormalTok{rating, mu\_hat)}
\end{Highlighting}
\end{Shaded}

\begin{verbatim}
## [1] 1.059951
\end{verbatim}

And any number will be higher than \(\hat{\mu}\) mean as we can see
below

\begin{Shaded}
\begin{Highlighting}[]
\NormalTok{predictions }\OtherTok{\textless{}{-}} \FunctionTok{rep}\NormalTok{(}\FloatTok{2.5}\NormalTok{, }\FunctionTok{nrow}\NormalTok{(test\_set))}
\FunctionTok{RMSE}\NormalTok{(test\_set}\SpecialCharTok{$}\NormalTok{rating, predictions)}
\end{Highlighting}
\end{Shaded}

\begin{verbatim}
## [1] 1.465849
\end{verbatim}

\hypertarget{movie-effect}{%
\paragraph{Movie effect}\label{movie-effect}}

Due to the large dataset for this project a linear regression would take
several time, a better option to work with is the least square estimate
of the movie effect \(\hat{b}_i\), it can be take from the average of
\(Y_{u,i}-\hat{\mu}\) in every movie \(i\). The following formula was
used to get the movie effect:

\[Y_{u,i}=\mu+b_i+\epsilon_{u,i}\]

And we can see the effect of the movies with the following plot:

\begin{figure}
\centering
\includegraphics{MovieLens-Report_files/figure-latex/movie-effect-1.pdf}
\caption{Distribution movie effect}
\end{figure}

The RMSE is lowest than the mean and it was a good improve, as it was
shown in data exploration some movies get more rated than others, we
still need improve more the result to get our objective.

\begin{Shaded}
\begin{Highlighting}[]
\CommentTok{\# Predict ratings adjusting for movie effects}
\NormalTok{predicted\_ratings}\OtherTok{\textless{}{-}}\NormalTok{mu\_hat}\SpecialCharTok{+}\NormalTok{test\_set}\SpecialCharTok{\%\textgreater{}\%}
  \FunctionTok{left\_join}\NormalTok{(movie\_avg, }\AttributeTok{by =} \StringTok{"movieId"}\NormalTok{) }\SpecialCharTok{\%\textgreater{}\%}
  \FunctionTok{pull}\NormalTok{(b\_i)}

\CommentTok{\# Calculate RMSE based on movie effects model}
\NormalTok{model\_1\_rmse}\OtherTok{\textless{}{-}}\FunctionTok{RMSE}\NormalTok{(test\_set}\SpecialCharTok{$}\NormalTok{rating, predicted\_ratings)}

\CommentTok{\#Save result and show difference between objective}
\NormalTok{df\_results}\OtherTok{\textless{}{-}}\FunctionTok{bind\_rows}\NormalTok{(df\_results,}
                      \FunctionTok{data.frame}\NormalTok{(}\AttributeTok{Method =} \StringTok{"Movie  Effect"}\NormalTok{, }\AttributeTok{RMSE =} \FunctionTok{as.character}\NormalTok{(}\FunctionTok{round}\NormalTok{(model\_1\_rmse,}\DecValTok{5}\NormalTok{)), }
                                 \AttributeTok{Difference =} \FunctionTok{as.character}\NormalTok{(}\FunctionTok{round}\NormalTok{(model\_1\_rmse,}\DecValTok{5}\NormalTok{)}\SpecialCharTok{{-}}\NormalTok{objective\_rmse)))}

\NormalTok{df\_results }\SpecialCharTok{\%\textgreater{}\%}\NormalTok{ knitr}\SpecialCharTok{::}\FunctionTok{kable}\NormalTok{()}
\end{Highlighting}
\end{Shaded}

\begin{longtable}[]{@{}lll@{}}
\toprule\noalign{}
Method & RMSE & Difference \\
\midrule\noalign{}
\endhead
\bottomrule\noalign{}
\endlastfoot
Objective & 0.8649 & - \\
Movie Effect & 0.94367 & 0.07877 \\
\end{longtable}

\hypertarget{user-effect}{%
\paragraph{User effect}\label{user-effect}}

Some users tend to rate more than others (quantity) and other are more
critical than others, so adding the user effect is slightly better than
only use the movie effect

\begin{figure}
\centering
\includegraphics{MovieLens-Report_files/figure-latex/user-effect-1.pdf}
\caption{Distribution user effect}
\end{figure}

\begin{longtable}[]{@{}lll@{}}
\toprule\noalign{}
Method & RMSE & Difference \\
\midrule\noalign{}
\endhead
\bottomrule\noalign{}
\endlastfoot
Objective & 0.8649 & - \\
Movie Effect & 0.94367 & 0.07877 \\
Movie + User Effect & 0.86561 & 0.00071 \\
\end{longtable}

\hypertarget{gender-effect}{%
\paragraph{Gender effect}\label{gender-effect}}

The movies acclaimed by critics then to be more rated by users thats why
we saw an improve using movies and users, but this happens with the
genre too. The table below shows the results.

\begin{longtable}[]{@{}lll@{}}
\toprule\noalign{}
Method & RMSE & Difference \\
\midrule\noalign{}
\endhead
\bottomrule\noalign{}
\endlastfoot
Objective & 0.8649 & - \\
Movie Effect & 0.94367 & 0.07877 \\
Movie + User Effect & 0.86561 & 0.00071 \\
Movie, User, Genre Effect & 0.86526 & 0.00036 \\
\end{longtable}

\hypertarget{year-effect}{%
\paragraph{Year effect}\label{year-effect}}

On the data exploration it was shown than it was a clearly year effect
reaching it highest peak on the 90's. It adds a modest improve of
0.00036.

\begin{longtable}[]{@{}lll@{}}
\toprule\noalign{}
Method & RMSE & Difference \\
\midrule\noalign{}
\endhead
\bottomrule\noalign{}
\endlastfoot
Objective & 0.8649 & - \\
Movie Effect & 0.94367 & 0.07877 \\
Movie + User Effect & 0.86561 & 0.00071 \\
Movie, User, Genre Effect & 0.86526 & 0.00036 \\
Movie, User, Genre, Year Effect & 0.86509 & 0.00019 \\
\end{longtable}

\hypertarget{date-review-effect}{%
\paragraph{Date Review effect}\label{date-review-effect}}

Our final bias is the date review, this column is a exact date with
hours, to get a better approximation it was rounded to week.

\begin{longtable}[]{@{}lll@{}}
\toprule\noalign{}
Method & RMSE & Difference \\
\midrule\noalign{}
\endhead
\bottomrule\noalign{}
\endlastfoot
Objective & 0.8649 & - \\
Movie Effect & 0.94367 & 0.07877 \\
Movie + User Effect & 0.86561 & 0.00071 \\
Movie, User, Genre Effect & 0.86526 & 0.00036 \\
Movie, User, Genre, Year Effect & 0.86509 & 0.00019 \\
Movie, User, Genre, Year, Date Review Effect & 0.8649 & 0 \\
\end{longtable}

This leads to the RMSE objective, but, can it be better?

\hypertarget{regularization-effect}{%
\paragraph{Regularization effect}\label{regularization-effect}}

Answering the last question, yes, we can improve the last result with
regularization. The following plot shows the RMSE tested for every
\(\lambda\) value tested. The optimal parameter is 4.7 which minimized
the RMSE to 0.86462.

\begin{figure}
\centering
\includegraphics{MovieLens-Report_files/figure-latex/lambdas-result-1.pdf}
\caption{Selecting best lambda}
\end{figure}

\begin{longtable}[]{@{}
  >{\raggedright\arraybackslash}p{(\columnwidth - 4\tabcolsep) * \real{0.7500}}
  >{\raggedright\arraybackslash}p{(\columnwidth - 4\tabcolsep) * \real{0.1053}}
  >{\raggedright\arraybackslash}p{(\columnwidth - 4\tabcolsep) * \real{0.1447}}@{}}
\toprule\noalign{}
\begin{minipage}[b]{\linewidth}\raggedright
Method
\end{minipage} & \begin{minipage}[b]{\linewidth}\raggedright
RMSE
\end{minipage} & \begin{minipage}[b]{\linewidth}\raggedright
Difference
\end{minipage} \\
\midrule\noalign{}
\endhead
\bottomrule\noalign{}
\endlastfoot
Objective & 0.8649 & - \\
Movie Effect & 0.94367 & 0.07877 \\
Movie + User Effect & 0.86561 & 0.00071 \\
Movie, User, Genre Effect & 0.86526 & 0.00036 \\
Movie, User, Genre, Year Effect & 0.86509 & 0.00019 \\
Movie, User, Genre, Year, Date Review Effect & 0.8649 & 0 \\
Regularized Movie, User, Genre, Year, Date Review Effect & 0.86462 &
-0.00028 \\
\end{longtable}

\hypertarget{final-testing}{%
\subsection{Final testing}\label{final-testing}}

With our algorithm defined and using using the training set and the test
set to validate the RMSE, for our final step, the date review was modify
to enhance the running time. The final result for the RMSE is 0.85621
that is less than the objective o the project:

\begin{Shaded}
\begin{Highlighting}[]
\NormalTok{df\_final\_model }\SpecialCharTok{\%\textgreater{}\%}\NormalTok{ knitr}\SpecialCharTok{::}\FunctionTok{kable}\NormalTok{()}
\end{Highlighting}
\end{Shaded}

\begin{longtable}[]{@{}lll@{}}
\toprule\noalign{}
Method & RMSE & Difference \\
\midrule\noalign{}
\endhead
\bottomrule\noalign{}
\endlastfoot
Objective & 0.8649 & - \\
Final RMSE & 0.85621 & -0.00869 \\
\end{longtable}

\hypertarget{conclusions}{%
\section{Conclusions}\label{conclusions}}

Analyzing the MovieLens dataset we found bias that were reduce to a RMSE
lower than the objective, this was thanks to regularization to.

We could improve this result using matrix factorization, singular value
descomposition (SVD) and principal component analysis (PCA). It quantify
residuals within this error loss based on patterns observed between
groups of movies or groups of users such that the residual error in
predictions can be further reduced.

\hypertarget{reference}{%
\section{Reference}\label{reference}}

Irizarry, Rafael A. 2020. Introduction to Data Science: Data Analysis
and Prediction Algorithms

with R. CRC Press.

\url{http://rafalab.dfci.harvard.edu/dsbook/}

\end{document}
